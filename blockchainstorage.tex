\documentclass[11pt]{article}

\usepackage{graphicx}
\usepackage{titling}
\usepackage[nottoc]{tocbibind}

\title{Verifying Diploma with Blockchain Technology}
\author{Kenneth Susort \and Christer Jensen}
\date{\today}


\setlength{\parindent}{3em}
\setlength{\parskip}{1em}

\graphicspath{ {./graphics/} }

\begin{document}

\begin{titlepage}
    \begin{center}
        
        \huge
        \textbf{\thetitle}
        
        \vspace{0.5cm}
        \large
       	How can the Bitcoin technology be used\\ for validating academic diplioms?
        
        \vspace{0.5cm}
        
        \textbf{\theauthor}
        
        \vfill
        
        A thesis presented for the degree of\\
        Bachelor in Computer Science
        
        \vspace{0.8cm}
        
        \includegraphics[width=0.4\textwidth]{university_logo.png}\\
        \large
        IDE\\
        University of Stavanger\\
        Norway\\
        \thedate
        
    \end{center}
\end{titlepage}

\begin{center}
	\vspace*{4cm}
	\textbf{\textit{Abstract}}
\end{center}


This is where the abstract goes. This is where the abstract goes. This is where the abstract goes. This is where the abstract goes. This is where the abstract goes. This is where the abstract goes. This is where the abstract goes. This is where the abstract goes. This is where the abstract goes. This is where the abstract goes.
\pagenumbering{roman}
\newpage

\tableofcontents

\newpage
\pagenumbering{arabic}

\section{Introduction}

Falsified diplomas is something alot of institutions struggle with. We propose a solution to this problem, by using blockchain technology to store hashes of diplomas in series. The hashes are then hashed together using a merkle tree hash structure. This allows one to store an unchangable hash that only the publisher of the original diploma would be able to prove ownership of. Attempts of changing stored documents will change every hash after said document, due to the blockchain tructure. Our goal is to design a so-called corsortium blockchain, which would use signatures from different institutions to validate a diploma, and locking it in the blockchain. 

\section{Background}
In the background section we will go over the different aspects and developments of cryptocurrency and the blockchain in general. We will also talk about the benefits and drawbacks of the variations of blockchains that exists, focusing on the more popular with the intent on finding what would suit validating documents best. 
\subsection{Bitcoin}
  
Bitcoin is a digital decentralized currency conceived in a paper by Satoshi Nakamoto\cite{nakamoto2009bitcoin}. Transactions works without an intermediary, like a bank, but rather peer-to-peer. Transactions are validated by nodes that updates the blockchain, which is a ledger containing every transaction made. These nodes offer computation power, or CPU cycles to record and verify the transactions in this ledger. This is called mining, and the nodes are called miners, because the reward for verifying and adding to the blockchain, is transaction fees for every transaction and the chance to unlock, or mine new bitcoins. For every new block in the blockchain that is validated, a set amount of bitcoins is made, and given to the miner who validated the block. This amount is decreasing by a set amount, and the difficulty of mining is increasing by total amount of mining power, so that it will take about 10 minutes between each successful mining.  
  
In the beginning, you could mine with your own personal computer, utilizing the CPU and be profitable. Then when more computers were utilized as miners, the difficulty increased so that it was less profitable to just use the CPU. Since mining is  simple hashing of a value, people figured out that graphical processor units, or GPUs, where better suited for mining than CPUs, and so the norm changed to using GPUs, which could produce a much greater rate of hashing. Then came the ASICs, which are dedicated mining machines and can produce even higher hashrates. Present day, miners gather their computational power in collaboration to increase their success rate, and sharing the mined bitcoins between every member, based on how much each individual contributed to the group. This is currently is the most profitable way of mining, though the electrical cost is catching up to the profitability of the current ASICs.

This method of validating transactions is called proof of work, and uses computational power in exchange for currency, is widely used in other cryptocurrencies as well as Bitcoin. Other proofs, or methods of validating transactions without a central bank or intermediary, has emerged since Bitcoin went public and became popular. Proof of stake, proof of burn, and proof of activity is examples of a few. These use different  types of approaches to validating and keeping track of transactions. Proof of stake needs miners to prove their ownership of their share in the currency. Peercoin, which implements proof of stake, uses coin age, or how long the coins have been held by a miner to increase chances of successfully validate the transaction ledger. Proof of burn requires miners to burn, or send part of their coins to an unusable address, for validating transactions and unlock new coins. 
	  
The popularity, and therefore its worth, of Bitcoin has significantly increased since its introduction in 2008. Since it’s controlled purely by supply and demand, the increased popularity has made the worth of a bitcoin gone from just a theoretical currency to a multi billion dollar trade, where the price of a single bitcoin has been as high as 1000 USD. After the bankruptcy of Mt. Gox, a bitcoin exchange handling about 70 percent of all transactions, where several hundred thousands of customers bitcoin where lost or stolen, the bitcoin took quite a hit, and hit the bottom at about 20 percent of what the value once was. Since then, the price has been steadily increasing again, and at present time is about 450 USD. With this increase in popularity, the cost of mining will surely continue to increase, and the need for cheaper alternatives to proof of work, or at least alterations, is increasing with the cost of mining.  

\subsection{The Blockchain}
The blockchain is by far the most fruitful idea that comes with bitcoin. It’s a linked list containing hashed blocks of transactions, linking to each other in a specific order. The hashing makes it very hard to alter anything on the blockchain, due to the fact that it is practically impossible to get the same hash values from different data. An individual or group with the intent of changing something in the blockchain would have to have 51 percent or more of all the computational power currently participating in the blockchain. This way, as long as there are more honest blockchain contributors than dishonest, the chain will not be compromised. If there is a difference in a block, or two blocks get solved at the same time, a fork will happen. This fork will continue till one of the competing chains is longer than the other, and the shortest will be annulled and discarded. 
  
Blocks in the blockchain is the building stone. A block is comprised of a header and a body. The header contains a version number, the hash of the previous block in the chain, a Merkle root of the transactions, the current difficulty, and the nonce. The previous block hash is there to link all the blocks together, so that if someone with ill intent changes something already in the blockchain, the hash values will differ and the block is dropped. The Merkle root, is a tree structure where the parents value is a hash of its children. So to check if two competing blocks are different, you would only have to check the Merkle root hash. The current difficulty of the blockchain, is the number of zeroes at the start of the hash. A nonce is a cryptographic term for a word or number used only once. This is the answer to the proof of work. The body contains all the transactions to be stored. These transactions are hashed only indirectly through the merkle root. Because of this, a block takes the same time to hash its transactions, independently of the number of transactions. 

\subsection{Different Blockchains}
	  
This design has grown in popularity also outside the cryptocurrency field, because of its ability to store something publicly without revealing what the data is. Industries which are researching ways to utilize the blockchain include, but not limited to, banks, stock exchange, document signing, realty, cloud storage, and many more. 

The original blockchain uses proof-of-work to validate transactions, by making a node spend computational power in exchange for the chance of mining a block. Bitcoin utilizes the Hashcash\cite{hashcash} proof-of-work system which works by hashing a base string and a nonce together, changing the nonce for each try, until the resulting hash is lower than the difficulty. An easier way to explain this is to count the amount of zeroes a hash starts with. The more zeroes, the lower the value, the more difficult it becomes to get. Because of the randomness that is SHA-256 hashing, it is very difficult to predict who will solve the proof-of-work first. Because of Bitcoins popularity, the Bitcoin blockchain has become increasingly difficult to mine. Different suggestions has come up to try and improve or replace this original SHA-256 proof-of work. 

Litecoin is the second most popular cryptocurrency as of yet. It is, as its name suggest, a lighter currency, compared to Bitcoin. Litecoin still uses proof-of-work as validation, but has made changes to overcome some of Bitcoins problems. Introducing scrypt, Litecoins proof-of-work system, utilizing more memory and makes it easier for normal computers to contribute and still make a profit. It still uses Bitcoins SHA-256, but as a subroutine. One of the drawbacks of scrypt is because it is easier for regular computers to mine, it is also more at risk for botnet exploitation or attacks. Big enough botnets have a greater chance of double-spending on a blockchain using scrypt than Bitcoins SHA-256 proof-of-work.

Moving away from proof-of-work, there are several other proof-of-x with different benefits and drawbacks. Proof-of-stake, first proposed on an online forum by a user named Quantum Mechanic, uses the currency itself as a stake to mining a new block. The way it works is by proving ownership of currency, and in return there is a chance to mine the next block. The more currency is held, the greater probability to unlock more. This is one of the big drawbacks with proof-of-stake, where the rich becomes richer. There are several implemented solutions to this, using randomization, coin age or movement. The most popular implementation of proof-of-stake is Peercoin, which utilizes randomization and coin age to achieve consensus. The probability of mining a block is calculated from the amount of coin, and how long the coin has remained in the specific wallet. Coins become able to compete for a block after 30 days, and will continue to increase in probability until it has gone 90 day, when the probability reaches a maximum value. Implementations based on movement, rewards coins that in exchanged more, with higher probability of finding the next block. 

One of proof-of-stake's biggest advantages and drawbacks is the cost. Because the currency is also the means of mining, there is little at stake outside the currency. Bitcoin mining uses on average 240kWh per bitcoin, which is a considerable cost, when you factor in the purchase cost of the ASICs that do the mining. The International Energy Agency states that for each megawatt(MW) of power spent, results in 650kg of CO2 released into the atmosphere. This means that for every bitcoin mined, 156kg CO2 is released. A proof-of-stake blockchain does not have this problem, but that also means that it is cheaper to exploit, by mining several forks at once. To combat this, Etherum, which is a blockchain for app development, suggested a system where you could punish anyone who was mining several blockchains by making them doublesign. This was never implemented because Etherum went with proof-of-work. 

Proof



\section{Citation Example}

I his recent book he mentions how to use citation\cite{TEST:1}. Also this article\cite{ART:3}. This is very different from how it's done in I Dont Cir\'es newest book\cite{TEST:2}. This article again\cite{ART:3}. \cite{nakamoto2009bitcoin}

\newpage
\bibliography{references} 
\bibliographystyle{ieeetr}

\end{document}